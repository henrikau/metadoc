\newpage
\section{MetaDoc Client API}
\label{sec:client_api}

The MetaDoc client calls a specified function based on the data passed between
client and server. Only one type of data should be sent per document, and both
the client and the server only checks for the expected type of data in the
recieved \gls{xml} document. 

Each data type has a named container element within the \gls{xml} document,
which there should only be one of per document. If a list of data is passed
between server and client, the list should be placed inside a container
element. The name of the container element is used in naming modules and
classes in order to ease readability of code. 

An example of such a container is \texttt{users}, which holds
\texttt{user\_entry} elements. An example of a MetaDoc \gls{xml} document with
a list of users is shown in listing \ref{lst:userlistexample}.

\scriptcode[lst:userlistexample]{User list XML example}{examples/xml/container.xml}{XML}

The naming of the class that handles the data passed between server and client
on the client side depends on whether data is passed from client to server, or
the other way around. 

\subsection{Sending data}
\label{sec:customizing_client_send}
Before data can be sent, the \gls{xml} document must be assembled. Because
there is not always a standard way to populate this data on every site, a
custom class is created. This class should be found under
\texttt{custom.site<name>.Site<Name>}, where \texttt{<name>} is the name of the
container element in the \gls{xml}, and \texttt{<Name>} is the name capitalized
(e.g.  \texttt{config} would use \texttt{custom.siteconfig.SiteConfig}). This
class should inherit from \\ \texttt{custom.abstract.MetaOutput}. 

When the client is ready to fetch these items, the function \texttt{populate()}
in this class is called. This function will gather the information to send,
creating elements found in
\texttt{<name>.definition.<Name>.legal\_element\_types} for this information
and placing these items in \texttt{self.items}. 

An example of such a class for the imaginary \texttt{posts} data is shown
below.

\scriptcode{custom.siteposts}{examples/metaelements/custom/siteposts.py}{python}

The client will then take care of packing data to \gls{xml}, sending the data
to the server, processing the receipt returned from the server and caching any
data that was not accepted by the server. For more information on caching, see
section \ref{sec:caching}.

\subsection{Recieving data}
\label{sec:customizing_client_recieve}
The client will take care of fetching the data from the server, unpacking the
\gls{xml} and creating objects based on the type of data recieved. Once this is
done, a function called \texttt{process()} will be called on the class
\texttt{custom.update<name>.Update<Name>}. When this function is called, the
object's \texttt{self.items} should be a list of \\
\texttt{<name>.definition.<Name>} objects.

Examples for producing files similar to the ones now in use based on
information transferred through MetaDoc is given in
\texttt{doc/examples/custom/}.

\subsection{Summary}
\texttt{process()} and \texttt{populate()} are the key functions that are used
to customize how the client handles data passed between the MetaDoc client and
server. These functions should be found on classes inside the \texttt{custom}
package in the MetaDoc client. 
