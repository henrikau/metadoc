\section{Using the MetaDoc client}
Usage of the MetaDoc client is done mainly through the use of \texttt{main.py}.
\texttt{main.py} takes care of sending and retrieving data to and from the 
server, caching any data that could not be sent, and validating XML data 
recieved.

When \texttt{main.py} should send data to the server, a custom function that
should populate the data to be sent is called, so that each site can customize
the way data is gathered on the site. 

When \texttt{main.py} recieves data from the server, it calls a custom function
based on the data recieved, where each site can define what should be done with
the recieved data.

Section \ref{sec:customizing_client} explains what functions are called and how
they should handle the data. 

\subsection{Configuration}
\label{sec:metadoc_conf}
The MetaDoc client uses a configuration file \texttt{metadoc.conf}. This file
\textit{must} be placed in the same folder as the client itself. 

The configuration is in INI format and defines a section named \texttt{MetaDoc}
which must contain the following parameters:

\begin{description}
    \item[host] \textbf{baseurl} for the MetaDoc Server that the client should
        communicate with.
    \item[key]  Absolute path to the clients SSL certificate key file. This
        should be the private key for \textbf{cert}, and is used to encrypt
        data passed to the server.
    \item[cert] Absolute path to the clients SSL certificate file. This is used
        to authenticate the client with the server. See section
        \ref{sec:authentication} for more information.
    \item[site\_name]   The name of the site the client is running on. 
\end{description}

The following parameters \textit{may} be defined:

\begin{description}
    \item[trailing\_slash]  Whether the client should append a trailing slash
        at the end of URLs used to connect to the server. At the moment, this
        should be set to \texttt{True}.
    \item[valid]    Initially set to \texttt{False} when \texttt{main.py}
        creates a sample configuration file. This is set to avoid running the
        client without being properly configured. If this value is present, it
        \textit{must} be set to \texttt{True} or \texttt{yes}.
\end{description}

\texttt{main.py} will create a initial configuration file if one is missing
when it starts. This can be used as a basis for a configuration file. 

\subsection{Handles}
\label{sec:handles}
\texttt{main.py} takes handles that tells the script what information to send
or retrieve to or from the server. All handles can be mixed together
\textit{except} for handles that override each other. Handles that overrides
others are explicitly stated below.

\texttt{main.py} takes the following handles:

\begin{description}
    \item[-h, --help]   Displays a short help message explaning the handles
    that may be passed to \texttt{main.py}. Overrides any other handles.
    \item[-v, --verbose]    Verbose mode. Prints information about progress and
    information sent and recieved between client and server. 
    \item[-q, --quiet]  Quiet mode. Prints nothing unless the program fails.
    Overrides \textbf{-v}, \textbf{--verbose}.
    \item[-l \textless log level\textgreater, --log-level=\textless log
    level\textgreater] Sets the log level for the program. See section 
    \ref{sec:logging} for more information about what is logged at different 
    levels.
    \item[-n, --no-cache]   Prevents the client from sending any cached data.
    For more information about caching, see section \ref{sec:caching}.
    \item[-e]   Sends event data from client to server.
    \item[-c]   Sends configuration data from client to server.
    \item[-s]   Sends software data from client to server.
    \item[-u]   Retrieves user data from the server.
    \item[-a]   Retrieves allocation data from server.
    \item[-p]   Retrieves project data from server.
    \item[--dry-run]    Does a dry run, not sending any data to server. Does
        not retrieve cached data, and does not save any data to cache.
        Should be run with verbose to see data that would be sent.
\end{description}

\subsection{Logging}
\label{sec:logging}
The client logs data to \texttt{/var/log/mapi/}. The folder must be read and
writable for the user that runs the client in order for the client to run.
The client creates a new log file each day it is run, with the name 
\texttt{metadoc.client.YYYY-mm-dd.log}. 

The client has five different logging levels. The list below gives an overview
of what is logged at the different levels. The higher items in the list contain
everything below as well, so that with a log level set to \textbf{error} will
also contain \textbf{critical} logging.

\begin{description}
    \item[debug]    Debugging information, used for development and error
    checking.
    \item[info] Information about what is happening during execution, such as
    items sent or recieved to/from the server.
    \item[warning]  Warnings occouring during execution, mainly problems that
    will not cause a failure but that should be addressed.
    \item[error]    Errors that cause partial failure of the execution, such as
    being unable to connect to the server.
    \item[critical] Critical failures that causes the execution to halt, or
    errors in the program code itself.
\end{description}

The log level defaults to the lowest possible, so everything will be logged if
nothing is set.

\subsection{Caching}
\label{sec:caching}
The client caches data to \texttt{/var/cache/mapi/}. Files are named after the
data type that is cached in each file. The user running the client must have
read and write access to the folder in order for the client to run. 

The client will cache any information that is not accepted by the server, 
\textit{unless} the server returns a reciept for the information that marks the 
information as invalid or malformed in some way, such that the information will 
not be accepted if resent at a later date. See section \ref{sec:errors} for
more information.

Data the client sends may be marked so that it will not resend any cached data
when the client is run with the same handle. This is mainly for use for full
updates, such as software and configuration, where any cached data would be
outdated or duplicates if sent together with a new run.

If the \textbf{-n} or \textbf{--no-cache} handles are passed, the script will
ignore any cached data completely and run as if it didn't exist. The cached
data will then be processed on the next run where \textbf{-n} or
\textbf{--no-cache} is not passed.

\subsection{Customizing the MetaDoc client}
\label{sec:customizing_client}

The MetaDoc client calls a specified function based on the data passed between
client and server. Only one type of data should be sent per document, and both
the client and the server only checks for the expected type of data in the
recieved XML document. 

Each data type has a named container element within the XML document, which
there should only be one of per document. If a list of data is passed between
server and client, the list should be contained within a container element. The
name of the container element is used in naming modules and classes in order to
ease readability of code. 

The naming of the class that handles the data passed between server and client
on the client side depends on whether data is passed from client to server, or
the other way around. For information about the classes and functions used when
sending data to the server, see section \ref{sec:customizing_client_send}, and
for data recieved from the server, see section
\ref{sec:customizing_client_recieve}. 

\subsubsection{Sending data}
\label{sec:customizing_client_send}
Data sent from the client to the server must first be populated. Because there
is not always a standard way to populate this data on every site, a custom
class is created. This class should be found under
\texttt{custom.site<name>.Site<Name>}, where \texttt{<name>} is the name of the
container element in the XML, and \texttt{<Name>} is the name capitalized (e.g. 
\texttt{config} would use \texttt{custom.siteconfig.SiteConfig}). This class
should inherit from \texttt{custom.abstract.MetaOutput}. 

When the client is ready to fetch these items, the function \texttt{populate()}
on this class will be called. This function should gather any information to be
sent, create elements found in
\texttt{<name>.definition.<Name>.legal\_element\_types} for this information 
and place these items in \texttt{self.items}. 

The client will then take care of packing data to XML, sending the data to the
server, processing the receipt returned from the server and caching any data
that was not accepted by the server. For more information on caching, see
section \ref{sec:caching}.

\subsubsection{Recieving data}
\label{sec:customizing_client_recieve}
Data that is recieved from the server must be processed. The client will take
care of fetching the data from the server, unpacking the XML and creating
objects based on the type of data recieved. Once this is done, a function
called \texttt{process()} will be called on the class
\texttt{custom.update<name>.Update<Name>}. When this function is called, the 
object's \texttt{self.items} should be a list of 
\texttt{<name>.definition.<Name>} objects.

Examples for producing files similar to the ones now in use based on
information transferred through MetaDoc is given in
\texttt{doc/examples/custom/}.
