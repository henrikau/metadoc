\newpage
\section{Using the MetaDoc client}
Usage of the MetaDoc client is done mainly through the use of \texttt{main.py}.
\texttt{main.py} takes care of sending and retrieving data from the server, as
well as caching any data that could not be sent. 

When \texttt{main.py} should send data to the server, a custom function that
should populate the data to be sent is called, so that each site can customize
the way data is gathered on the site. 

When \texttt{main.py} recieves data from the server, it calls a custom function
based on the data recieved, where each site can define what should be done with
the recieved data.

\subsection{Handles}
\label{sec:handles}
\texttt{main.py} takes handles that tells the script what information to send
or retrieve to or from the server. All handles can be mixed together
\textit{except} for handles that override each other. Handles that overrides
others are explicitly stated below.

\texttt{main.py} takes the following handles:

\begin{description}
    \item[-h, --help]   Displays a short help message explaning the handles
    that may be passed to \texttt{main.py}. Overrides any other handles.
    \item[-v, --verbose]    Verbose mode. Prints information about progress and
    information sent and recieved between client and server. 
    \item[-q, --quiet]  Quiet mode. Prints nothing unless the program fails.
    Overrides \textbf{-v}, \textbf{--verbose}.
    \item[-l \textless log level\textgreater, --log-level=\textless log
    level\textgreater] Sets the log level for the program. See section 
    \ref{sec:loglevels} for more information about what is logged at different 
    levels.
    \item[-n, --no-cache]   Prevents the client from sending any cached data.
    For more information about caching, see section \ref{sec:caching}.
    \item[-e]   Sends event data from client to server.
    \item[-c]   Sends configuration data from client to server.
    \item[-s]   Sends software data from client to server.
    \item[-u]   Retrieves user data from the server.
    \item[-a]   Retrieves allocation data from server.
    \item[-p]   Retrieves project data from server.
    \item[--dry-run]    Does a dry run, not sending any data to server. Should
    be run with verbose to see data that would be sent.
\end{description}

\subsection{Log levels}
\label{sec:loglevels}
The client has five different logging levels. The list below gives an overview
of what is logged at the different levels. The higher items in the list contain
everything below as well, so that with a log level set to \textbf{error} will
also contain \textbf{critical} logging.

\begin{description}
    \item[debug]    Debugging information, used for development and error
    checking.
    \item[info] Information about what is happening during execution, such as
    items sent or recieved to/from the server.
    \item[warning]  Warnings occouring during execution, mainly problems that
    will not cause a failure but that should be addressed.
    \item[error]    Errors that cause partial failure of the execution, such as
    being unable to connect to the server.
    \item[critical] Critical failures that causes the execution to halt, or
    errors in the program code itself.
\end{description}

The log level defaults to the lowest possible, so everything will be logged if
nothing is set.

\subsection{Caching}
\label{sec:caching}
The client will cache any information that is not accepted by the server, 
\textit{unless} the server returns a reciept for the information that marks the 
information as invalid or malformed in some way, such that the information will 
not be accepted if resent at a later date. See section \ref{sec:errors} for
more information.

Data the client sends may be marked so that it will not resend any cached data
when the client is run with the same handle. This is mainly for use for full
updates, such as software and configuration, where any cached data would be
outdated or duplicates if sent together with a new run.

If the \textbf{-n} or \textbf{--no-cache} handles are passed, the script will
ignore any cached data completely and run as if it didn't exist. The cached
data will then be processed on the next run where \textbf{-n} or
\textbf{--no-cache} is not passed.

\subsection{Customizing MetaDoc}
\label{sec:customizing_client}

\subsubsection{Sending data}
To send data to the server, the client creates instances of a sub-class of the
\\
\texttt{custom.abstract.MetaOutput} class. These classes should define a
\texttt{populate()} function that populates \texttt{self.items} with a set of
\texttt{metaelement.MetaElement} sub-class instances. Once \texttt{self.items}
is populated, the server packs it to XML and sends it to the server.

The server \textit{must} return a reciept for each entry, specifying whether
the entry has been accepted by the server, or return an error code, as defined
i table \ref{tbl:server_error_codes}, if the entry is not accepted. See
section \ref{sec:errors} and \ref{sec:caching} for more information.

\subsubsection{Recieving data}
When the client recieves data from the server, it parses the data and places
the parsed data in a \texttt{custom.abstract.MetaInput} instance. Which
\texttt{MetaInput} sub-class is used is defined in the element's description by
the class variable \texttt{update\_handler}.

Examples for producing files similar to the ones now in use based on
information transferred through MetaDoc is given in
\texttt{doc/examples/custom/}.
