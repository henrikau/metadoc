\newpage
\section{MetaDoc Server API}
\label{sec:server_api}

The MetaDoc server implements a REST-like API, however, there are certain
differences from REST noted in section \ref{sec:diff_from_rest}.

When the client performs a GET request on an availible URL, the server should 
return an XML document, or a HTTP status code refering to an error. 
The XML document should follow the MetaDoc DTD \cite{metadoc_dtd}. Each URL
only returns data from the requested data type. This means that a request to
\textbf{baseurl/allocations/} will return an MetaDoc XML document containing
only an \texttt{<allocations>} directly on the \texttt{<MetaDoc>} root, with
\texttt{<all\_entry>} tags as children of \texttt{<allocations>}. The client
should disregard any information outside \texttt{<allocations>} when connecting
to \textbf{baseurl/allocations/}. 

In order to send data to the server, the client performs a POST request, with
the POST data variable \texttt{metadoc} containing a MetaDoc XML document. The
server will only accept data from the data type specified in the URL, and will
disregard any other information. This means that a POST to
\textbf{baseurl/events/} should be a MetaDoc XML document containing a 
\texttt{<events>} tag directly on the \texttt{<MetaDoc>} root, with any number
of \texttt{<resourceUp>} and \texttt{<resourceDown>} tags as children of
\texttt{<events>}. 

When this data is sent to the server, the server should return a MetaDoc XML
document containing a \texttt{<receipts>} tag, with a \texttt{<r\_entry>} tag
for each element recieved that has an \textbf{id}-attribute. 

\subsection{Availible URLs}

\begin{description}
    \item[baseurl/allocations/] Retrieves a list of allocations/quotas relevant
        to the client
    \item[baseurl/users/] Retrieves a list of users for the client
    \item[baseurl/projects/] Retrieves a list of projects relevant to the
        client
    \item[baseurl/config/] Sends system configuration to server
    \item[baseurl/events/] Sends site events to the server
    \item[baseurl/software/] Sends system software to server
\end{description}

\subsection{Authentication}
\label{sec:authentication}
The site uses SSL certificates to authenticate the client. In order for the
site to be authenticated properly, the server must be aware of the client's
certificate already, and the correct owner of the certificate must be saved on
the server. This \textit{must} be the same as the value for \texttt{site\_name}
set in the MetaDoc configuration (see section \ref{sec:metadoc_conf}).

\subsection{Differences from REST}
\label{sec:diff_from_rest}

There are certain differences in the API compared to the REST specification. The 
MetaDoc Server API makes use of HTTP POST where HTTP PUT should be used in 
accordance with REST. This is due to limitations in standard Python libraries.

Because the access the MetaDoc Server API gives to the client is limited, this 
change does not prohibit any other functionality. 

No address currently supports both adding and retrieving data. This is not a
limitation in the system itself, but a matter of authorative sources of
information. 

\subsection{Server HTTP responses}

The server makes use of HTTP status codes to identify what error has occoured
if the server is unable or unwilling to process the request from the server. 

If the client does not send a SSL certificate, sends a sertificate unknown to 
the server, attempts to get information about sites not identified with the
certificate, or attempts to send information that identifies as another site, 
the server returns a ``403 Forbidden`` status code.

If the server fails to process the request, it will return a ``500 Server
Error`` status code. This does \textit{not} include errors on the document that
results in returning receipts.
