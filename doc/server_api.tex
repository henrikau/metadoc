\newpage
\section{MetaDoc API description}

\subsection{Server API}

The MetaDoc server implements a REST-like API. The server defines several URLs 
that can be accessed from the client:

\begin{description}
    \item[baseurl/allocations/] Retrieves a list of allocations relevant to the 
        site
    \item[baseurl/users/] Retrieves a list of users for the site
    \item[baseurl/projects/] Retrieves a list of projects relevant to the site
    \item[baseurl/config/] Sends system configuration to server
    \item[baseurl/events/] Sends site events to the server
    \item[baseurl/software/] Sends system software to server
\end{description}

When sending information to the MetaDoc Server, only the information relevant to 
that URL is processed. Any XML data sent that is not relevant for that URL is 
discarded, e.g. event information sent to \textbf{/baseurl/config/} will be 
discarded by the server. No receipt will be returned for this data.  

The server will return a MetaDoc XML document containing a \texttt{<reciept>} 
element, which will contain \texttt{<r\_entry>} elements for each element 
recieved. The \texttt{<r\_entry>} element should return a code from table 
\ref{tbl:server_error_codes} for each element. See section \ref{sec:errors} for 
more information on errors. 

\subsubsection{Differences from REST}

There are certain differences in the API compared to the REST specification. The 
MetaDoc Server API makes use of HTTP POST where HTTP PUT should be used in 
accordance with REST. This is due to limitations in standard Python libraries.

Because the access the MetaDoc Server API gives to the client is limited, this 
change does not prohibit any other functionality. 

\subsubsection{Server HTTP responses}

The server makes use of HTTP status codes.

If the client does not send a SSL certificate, sends a sertificate unknown to 
the server, or attempts to get information about sites not identified with the
certificate, the server returns a ``403 Forbidden`` status code.
