\newpage
\section{XML document}
\label{sec:xmldoc}

The XML document should follow the form described in the MetaDoc DTD 
\cite{metadoc_dtd}. 

\subsection{Document build}

Any type of information sent should only create one direct child of the root
element, \texttt{<MetaDoc>}. This means that when lists of information is sent, 
the list elements should be placed within a container element, and \textit{not} 
directly in the root element.

An example is that \texttt{<user\_entry>} elements are placed within a
\texttt{<users>} element. 

\subsection{Element attributes}

All element attributes \textit{must} be strings. This is because the attributes 
must be placed in the XML document, and without knowing the way to represent the 
attribute as a string it is not possible to properly use it as one.

After the attributes \texttt{clean}-function is run, it will be checked that
the attribute value is a \textbf{basestring} (\textbf{unicode} or
\textbf{str}). If any attribute is not, the element will not be sent. 

\subsection{Dates}

All dates in the document should be on the form specified by RFC3339 
\cite{rfc3339}. The \texttt{utils} module provides a function 
\texttt{date\_to\_rfc3339} that takes a \texttt{datetime.datetime} object and 
returns a string on RFC3339 form. It also provides a function
\texttt{rfc3330\_to\_date} which will return a \texttt{datetime.datetime}
object from a proper RFC3339 string, or \textbf{False} if the string is not a
correct RFC3339 date.

\subsection{Special attributes}

The \textbf{id} attribute of elements have a special function in MetaDoc. This 
attribute is used to identify the object when recieving receipts from the server 
whether elements have been added. The attribute is \textit{not} saved in caching 
to avoid duplicate \textbf{id}s when resending cached data together with new 
data. If you want to give elements a special identifier that should be saved, it 
must be called something other than \textbf{id}.

\subsection{Client}

\subsubsection{Cleaning attributes}

When an element is added as a sub-element to \texttt{metaelement.MetaElement}, a 
clean function is called for each attribute. The element defenition for 
sub-elements added may implement a function called 
\texttt{clean\_<attribute name>}. This function should validate that the value 
of the attribute and make sure it returns the string value of the attribute. 
This makes it possible to create elements by passing non-string variables, such 
as \texttt{datetime} objects for date fields, then converting them in the clean 
function. 
